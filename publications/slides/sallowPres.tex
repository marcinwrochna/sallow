\documentclass[aspectratio=169]{beamer}
\usepackage{microtype}
\usepackage{enumitem}
\usepackage{tikz}
\usepackage{amsmath,amssymb,amsfonts}
\usepackage{xspace}

\usepackage{lato}
%\usepackage{fontspec}

\usetikzlibrary{decorations.pathreplacing,decorations,calc,bending}

\usepackage{pifont}% http://ctan.org/pkg/pifont
\newcommand{\cmark}{\ding{51}}%
\newcommand{\xmark}{\ding{55}}%

\newtheorem{conjecture}{Conjecture}

\title[Sallow]
{Sallow: a heuristic algorithm for treedepth decompositions}
\author%[Author, Anders] (optional, for multiple authors)
{Marcin Wrochna}
\institute[Oxford] % (optional)
{
	\quad Durham University \qquad\qquad\qquad
%	\inst{1}%
	University of Oxford, UK
%	\and
%	\inst{2}%
%	University of Warsaw, Poland
}
\date[]{}
\subject{Theoretical Computer Science}

\newcommand\vv[1]{\mathbf{#1}}
\def\x{\vv{x}}
\def\y{\vv{y}}
\def\z{\vv{z}}
\def\b{\vv{b}}
\DeclareMathOperator{\supp}{supp}
\DeclareMathOperator{\td}{td}
\DeclareMathOperator{\pw}{pw}
\DeclareMathOperator{\poly}{poly}
\DeclareMathOperator{\tw}{tw}
\DeclareMathOperator{\cc}{cc}
\DeclareMathOperator{\ar}{ar}
\DeclareMathOperator{\opt}{opt}
\DeclareMathOperator{\Bx}{Box}
\DeclareMathOperator{\Struct}{Struct}
\newcommand\AOver[2]{\Struct^{#1}_{#2}}
\def\tdD{\td_D}
\def\Gg{\mathcal{G}}
\def\Cc{\mathcal{C}}
\def\Aa{\mathcal{A}}
\def\RR{\mathbb{R}}
\def\AA{\mathbb{A}}
\def\QQ{\mathbb{Q}}
\def\DD{\mathbb{D}}
\def\XX{\mathbb{X}}
\def\BB{\mathbb{B}}
\def\CC{\mathbb{C}}
\def\ZZ{\mathbb{Z}}
\def\eps{\varepsilon}
\def\dopt{\mathrm{d}_{\opt}}
\def\Sphere{\mathcal{S}}


\definecolor{dR}{RGB}{191,0,0}
\definecolor{dG}{RGB}{183,255,69}
\definecolor{dB}{RGB}{93,129,255}
\definecolor{dY}{RGB}{255,239,114}
\definecolor{dV}{RGB}{116,0,181}
\tikzset{ %
	B/.style={circle,fill=black!60!blue,inner sep=0pt,minimum size=7pt},
	W/.style={circle,fill=white,draw=black,inner sep=0pt,minimum size=7pt},
}	



\tikzset{
	v/.style={circle,draw=black!75,inner sep=0pt,minimum size=5pt}
}

\newcommand\auth[1]{\textcolor{teal}{#1}}
\newcommand\vs{\textrm{vs}\xspace}

\beamertemplatenavigationsymbolsempty

\definecolor{myblue}{RGB}{8,64,138}
\definecolor{pace}{RGB}{138,62,0}


\setbeamercolor{frametitle}{fg=myblue!70!black!80,bg=myblue!6}
\setbeamerfont{frametitle}{size=\Large,family=\flafamily}


\begin{document}
	

%\begin{frame}
%	\centering
%	{\Large\  Graph structure that allows approximating\\Max Constraint Satisfaction Problems}
%\end{frame}
	
%\usebackgroundtemplate{\vbox to \paperheight{\vfil\hbox to \paperwidth{\includegraphics[width=\paperwidth]{bg2.png}}}}%
{
\setbeamercolor{background canvas}{bg=gray!10}
\begin{frame}
	%\titlepage
	\ \\
	{\bfseries\color{pace} PACE 2020}\\[0.4ex]
	{\color{pace}\small\textsf{Parameterized Algorithms \& Computational Experiments Challenge}}\\[3.5em]
		
	{\fontsize{16}{19}\bfseries\color{myblue} Sallow}\\[0.4ex]
	{\color{myblue}a heuristic algorithm for treedepth decompositions}\\[2.5em]
	
	{\flafamily\small Marcin Wrochna}\\[0.4ex]
	{\flafamily\small University of Oxford, UK}\\[3em]
\end{frame}
}


\begin{frame}[t]
\only<1-9>{\frametitle{Greedy elimination}}
\only<10-13>{\frametitle{Greedy building}}
\only<14->{\frametitle{Greedy building + lookup}}
\vspace*{-0.8\baselineskip}
\begin{center}
	\tikzset{
		v/.style={draw=black!70, fill=green!30!gray!90,  circle, inner sep=0pt, minimum size=9pt},
		vsel/.style={v, fill=red!70!black},
		xx/.style={v, draw=black!50, fill=yellow!80!gray!10!white},
		e/.style={thick, black!70},
		g/.style={thick,white},
		gy/.style={g,yellow!50!white}
	}

	\pgfdeclarelayer{bg}
	\pgfsetlayers{bg,main}	
	\begin{tikzpicture}[scale=1.2,font=\large]
		\def\wid{7}
		\def\hei{5}
		\draw[white,thin] (-3.8,-5.0) -- (6.2,1.5);		
		\def\rr{0.4}
		\begin{scope}[yscale=0.45]
			\pgfmathsetseed{2}
			\draw[white, fill=black!50!yellow!40!white]
				plot [smooth cycle, samples=30,domain={0:360*(1-1./30)}] (-\x:2.5+0.7*rnd);
			\node[xx] (x1)  at (0.2,-0.5) {};
			\only<-11>{
				\node[xx] (x2)  at (-0.8,-2) {};			
			}
			\node[xx] (x3)  at (-2,-1) {};						
			\node[xx] (x4)  at (-1.2,0) {};
			\node[xx] (x5)  at (1.2,0.4) {};									
			\node[xx] (x6)  at (0.8,-2.2) {};			
			\node[xx] (x7)  at (0.8,-2.2) {};	
			\node[xx] (x8)  at (2.0,-1.8) {};
			\node[xx] (x9)  at (-0.3,1.5) {};			
		\end{scope}
		\only<-13>{
		\draw[g] (x3) to ++(100:\rr);
		\draw[g] (x3) to ++(150:\rr);
		\draw[g] (x3) to ++(-30:\rr);			
		\draw[g] (x4) to ++(45:\rr);
		\draw[g] (x4) to ++(170:\rr);
		\draw[g] (x4) to ++(-70:\rr);	
		\draw[g] (x2) to ++(80:0.8*\rr);
		\draw[g] (x2) to ++(173:0.8*\rr);
		\draw[g] (x2) to ++(10:0.8*\rr);				
		\draw[g] (x1) to ++(30:\rr);
		\draw[g] (x1) to ++(130:\rr);
		\draw[g] (x1) to ++(-120:\rr);			
		\draw[g] (x1) to ++(-50:\rr);			
		\draw[g] (x7) to ++(160:\rr);
		\draw[g] (x7) to ++(40:\rr);
		\draw[g] (x5) to ++(120:\rr);			
		\draw[g] (x5) to ++(-160:\rr);
		\draw[g] (x5) to ++(-10:\rr);						
		\draw[g] (x8) to ++(140:\rr);
		\draw[g] (x8) to ++(80:\rr);						
		\draw[g] (x9) to ++(170:\rr);
		\draw[g] (x9) to ++(30:\rr);		
		\draw[g] (x9) to ++(-80:\rr);		
		}

		\only<1-4,10->{
				
		\begin{scope}[shift={(-2,-2)}]
			\draw[white, fill=green!20!gray!50]
				plot [smooth cycle] coordinates {(0,0) (1.2,-2.3) (-1.2,-2.3)};
			\node[v] (v1)  at (0,-0.3) {};
			\node[v] (v2)  at (-0.3,-0.9) {};			
			\node[v] (v3)  at ( 0.3,-0.9) {};						
			\node[v] (v4)  at (-0.7,-1.5) {};						
			\node[v] (v5)  at (-0.25,-1.5) {};						
			\node[v] (v6)  at ( 0.7,-1.5) {};						
			\node[v] (v7)  at ( 0.25,-1.5) {};									
			\node[v] (v8)  at (-0.6,-2.1) {};						
			\node[v] (v9)  at (-0.1,-2.1) {};									
			\node[v] (v10) at (0.9,-2.1) {};						
			\node[v] (v11) at (0.4,-2.1) {};										
			\draw[e,->] (v11)--(v6);
			\draw[e,->] (v10)--(v6);
			\draw[e,->] (v9)--(v5);
			\draw[e,->] (v8)--(v5);
			\draw[e,->] (v7)--(v3);
			\draw[e,->] (v6)--(v3);
			\draw[e,->] (v5)--(v2);
			\draw[e,->] (v4)--(v2);
			\draw[e,->] (v3)--(v1);
			\draw[e,->] (v2)--(v1); 
		\end{scope}
		
		\begin{scope}[shift={(0.2,-2.4)}]
			\draw[white, fill=green!20!gray!50]
				plot [smooth cycle] coordinates {(0,0) (1.2,-2.3) (-0.2,-2.3) (-0.7,-1.15)};
			\node[v] (u1)  at (-0.07,-0.3) {};
			\node[v] (u2)  at (-0.3,-0.9) {};			
			\node[v] (u3)  at ( 0.3,-0.9) {};						
			\node[v] (u5)  at (-0.4,-1.5) {};						
			\node[v] (u6)  at ( 0.7,-1.5) {};						
			\node[v] (u7)  at ( 0.25,-1.5) {};									
			\node[v] (u9)  at (-0.1,-2.1) {};									
			\node[v] (u10) at (0.9,-2.1) {};						
			\node[v] (u11) at (0.4,-2.1) {};										
			\draw[e,->] (u11)--(u7);
			\draw[e,->] (u10)--(u7);
			\draw[e,->] (u9)--(u5);
			\draw[e,->] (u7)--(u3);
			\draw[e,->] (u6)--(u3);
			\draw[e,->] (u5)--(u2);
			\draw[e,->] (u3)--(u1);
			\draw[e,->] (u2)--(u1); 
		\end{scope}	
		
	\begin{scope}[shift={(2.2,-2.1)}]
			\draw[white, fill=green!20!gray!50]
				plot [smooth cycle] coordinates {(0,0) (0.5,-0.8) (0.4,-1.7) (-1,-1.7)};
			\node[v] (w1)  at (0.04,-0.3) {};
			\node[v] (w2)  at (-0.3,-0.9) {};			
			\node[v] (w3)  at ( 0.3,-0.9) {};						
			\node[v] (w4)  at (-0.7,-1.5) {};						
			\node[v] (w5)  at (-0.25,-1.5) {};												
			\node[v] (w7)  at ( 0.25,-1.5) {};
			\draw[e,->] (w7)--(w3);
			\draw[e,->] (w5)--(w2);
			\draw[e,->] (w4)--(w2);
			\draw[e,->] (w3)--(w1);
			\draw[e,->] (w2)--(w1); 
		\end{scope}			
	}

	\only<1> {
		\draw[e,->] (v1) to[out=90,in=-90] (x4);
		\draw[e,->] (v1) to[out=95,in=-85] (x3);		
		\draw[e,->] (v4) to[out=90,in=-90] (x3);		
		\draw[e,->] (v6) to[out=113,in=-85] (x4);			
		\draw[e,->] (v6) to[bend left] (x2);	
		\draw[e,->] (u2) to[out=90,in=-90] (x1);						
		\draw[e,->] (u3) to[out=90,in=-90] (x2);								
		\draw[e,->] (u6) to[bend left=20] (x7);								
		\draw[e,->] (w1) to[out=90,in=-90] (x5);
		\draw[e,->] (w2) to[out=70,in=-50] (x7);				
		\draw[e,->] (w7) to[out=45,in=-80] (x8);		
	}
	\only<2> {
		\draw[e,->,very thick] (u6) to[bend left=20] (x7);									
		\draw[e,->,dashed,black!30!red,very thick] (u1) to[bend left=20] (x7);										
	}
	\only<3> {
		\draw[e,->,very thick] (u1) to[bend left=20] (x7);										
	}	
	\only<4,5> {
		\draw[e,->] (v1) to[out=90,in=-90] (x4);
		\draw[e,->] (v1) to[out=95,in=-85] (x3);		
		\draw[e,->] (v1) to[out=90,in=-90] (x2);
		\draw[e,dashed,black!30!red,very thick] (x3)--(x4)--(x2)--(x3);
	}		
	\only<5-9>{
		\begin{scope}[shift={(-2,-2)},xscale=0.8]
			\node[vsel] (vv1)  at (0,-0.3) {};
			\node[v,minimum size=6pt] (vv1)  at (1,-0.6) {};			
			\node[v,minimum size=5pt] (vv1)  at (2,-0.8) {};						
			\node[v,minimum size=4pt] (vv1)  at (3,-0.94) {};						
			\node[v,minimum size=3pt] (vv1)  at (4,-1) {};						
			\node[v,minimum size=3pt] (vv1)  at (5,-1) {};						
			\node[v,minimum size=3pt] (vv1)  at (6,-1) {};			
		\end{scope}		
	}
	\only<6-9>{
		\begin{scope}[shift={(-0.8,-1.1)}]			
			\draw[white,fill=white] plot [smooth cycle] coordinates {
				(-270:0.6) (-180:0.6) (-160:1.6) (-20:1.6) (0:0.6)
			};
		\end{scope}
		\begin{scope}[shift={(-2,-2)},xscale=0.8]
			\node[v] (vv1)  at (0,-0.3) {};
		\end{scope}		
		\node[vsel] (x2)  at (-0.8,-1.2) {};	
		\draw[e,->] (x2) to[out=90,in=-90] (x3);
		\draw[e,->] (x2) to[out=90,in=-90] (x4);
		\draw[e,->] (x2) to[out=90,in=-90] (x9);	
		\draw[e,dashed,black!30!red,very thick] (x3)--(x4)--(x9) to[bend right] (x3);			
	}
	\only<7-8>{
		\node[align=center] at (4.2, 0.4)  {\only<8>{$\alpha \cdot {}$} degree\\at elimination\\[4pt]\Large$+$};
		\node[align=center] at (4.2,-0.8) {\only<8>{\quad $\beta \cdot {}$}  height of tree below\\ $h_{\circ} := \max(h_{\circ}, 1+h_{\textcolor{red}{\bullet}})$};
	}
	\only<9>{
		\node[align=center] at (4.2,0) {$\sim nd^2$ edges\\ :( };
	}
		
	\only<1-6,10-12> {
		\node at (4.2,0) {unprocessed};
	}
	%\only<10->{
	%	\node[align=center] at (4.2,0) {unprocessed\\ind. subgraph};
	%}	
	\only<-12>{	
		\node at (4.2,-3) {processed};		
	}
	\only<13-15>{
		\node[align=center] at (4.2,-3) {find-union\\structure};			
	}
	\only<13>{	
		\node[align=center] at (4.2,0) {unprocessed\\(static)};
	}	
	\only<14>{		
		\node[align=center] at (4.2,-0.5) {update score\\for a few best\\on the heap};			
	}	
	\only<15-17>{		
		\node[align=center] at (4.2,-0.5) {update score\\for top of heap\\1000 times};			
	}		
		
	\only<10>{
		\draw[e,->] (v1) to[out=90,in=-90] (x4);
		\draw[e,->] (v1) to[out=90,in=-90] (x3);		
		\draw[e,->] (v1) to[out=90,in=-90] (x2);
		\draw[e,->] (u1) to[out=95,in=-90] (x1);						
		\draw[e,->] (u1) to[out=99,in=-81] (x2);								
		\draw[e,->] (w1) to[out=90,in=-90] (x5);
		\draw[e,->] (w1) to[out=90,in=-90] (x7);				
		\draw[e,->] (w1) to[out=90,in=-90] (x8);	
	}
	\only<11-12>{
		\begin{scope}[shift={(-0.8,-1.1)}]			
			\draw[white,fill=white] plot [smooth cycle] coordinates {
				(-270:0.6) (-180:0.6) (-160:1.6) (-20:1.6) (0:0.6)
			};
		\end{scope}
		\begin{scope}[shift={(-2,-2)},xscale=0.8]
			\node[v] (vv1)  at (0,-0.3) {};
		\end{scope}	
		\node[vsel] (x2)  at (-0.8,-1.3) {};		
		\draw[e,->] (w1) to[out=90,in=-90] (x5);
		\draw[e,->] (w1) to[out=90,in=-90] (x7);				
		\draw[e,->] (w1) to[out=90,in=-90] (x8);			
	}
	\only<11>{
		\draw[e,->] (x2) to[out=90,in=-90] (x3);
		\draw[e,->] (x2) to[out=90,in=-90] (x4);
		\draw[e,->] (x2) to[out=90,in=-90] (x9);		
		\draw[e,->] (v1) to[out=90,in=-90] (x4);
		\draw[e,->] (v1) to[out=90,in=-90] (x3);		
		\draw[e,->] (v1) to[out=90,in=-90] (x2);
		\draw[e,->] (u1) to[out=95,in=-90] (x1);						
		\draw[e,->] (u1) to[out=99,in=-81] (x2);								
	
	}
	\only<12>{
		\draw[e,->] (x2) to[out=90,in=-90] (x3);
		\draw[e,->] (x2) to[out=90,in=-90] (x4);
		\draw[e,->] (x2) to[out=90,in=-90] (x9);			
		\draw[e,->] (v1) to (x2);
		\draw[e,->] (x2) to[out=95,in=-90] (x1);						
		\draw[e,->] (u1) to (x2);								
		%\node[v] at (-0.8,-1) {1}; \node[v] at (-1.3,-1.3) {2};	\node[v] at (-2.7,-2.3) {3}; \node[v] at (-3.4,-4.3) {4}; \node[v] at (1.6,-4.6) {5}; \node[v] at (-0.3,-1.3) {0};				
		\draw[green!40!gray!50, ultra thick]
			plot [smooth cycle] coordinates {(-0.9,-1) (-1.3,-1.3) (-2.6,-2.3) (-3.3,-4.5) (1.5,-4.8) (-0.3,-1.3)};
	}
	\only<13-17>{
		\node[vsel] (x2)  at (-0.8,-1) {};	
		\draw[e,->] (v1) to[out=90,in=-90] (x4);
		\draw[e,->] (v1) to[out=90,in=-90] (x3);		
		\draw[e,->] (v1) to[out=90,in=-110] (x2);
		\draw[e,->] (u1) to[out=95,in=-90] (x1);						
		\draw[e,->] (u1) to[out=99,in=-70] (x2);								
		\draw[e,->] (w1) to[out=90,in=-90] (x5);
		\draw[e,->] (w1) to[out=90,in=-90] (x7);				
		\draw[e,->] (w1) to[out=90,in=-90] (x8);		
	
		\draw[e,->,dotted,black!50!red,ultra thick] (x2) to[out=-90,in=65] (v6);
		\draw[e,->,dotted,black!50!red,ultra thick] (x2) to[out=-90,in=150] (u9);
		\draw[e,->,black!40!red,very thick] (v6) to (v3);
		\draw[e,->,black!40!red,very thick] (u9) to (u5);
		\draw[e,->,black!40!red,very thick] (v3) to (v1);		
		\draw[e,->,black!40!red,very thick] (u5) to (u2);		
		\draw[e,->,black!40!red,very thick] (u2) to (u1);					
	}
	\only<14-16>{
		%\draw[e,black!40!red] (x3)--(x4)--(x2)--(x3) (x2)--(x1);
		\draw[e,->,black!40!red,very thick] (x2) to[out=90,in=-90] (x3);
		\draw[e,->,black!40!red,very thick] (x2) to[out=90,in=-90] (x4);
		\draw[e,->,black!40!red,very thick] (x2) to[out=90,in=-90] (x9);			
		\draw[e,->,black!40!red,very thick] (x2) to[out=95,in=-90] (x1);		
	}	
	\only<16-17>{
	 	\node[align=left, fill=white] at (4.2,-2.2) {
	 		72\% in 16 min\\
			86\% in 3000 min\\
			96\% w. FlowCutter
		};
	}
	\only<17>{
		\node at (4.5,-4) {\LARGE\textcolor{gray}{Thanks!}};	
		
	\iftrue
		\draw[e,->,black!40!red,very thick] (v3) to (v1);	
		\draw[e,->,black!40!red,very thick] (v4) to (v2);
		\draw[e,->,black!40!red,very thick] (v9) to (v5);		
		\draw[e,->,black!40!red,very thick] (v10) to (v6);		
		\draw[e,->,black!40!red,very thick] (v11) to (v6);				

		\draw[e,->,black!40!red,very thick] (u2) to (u1);			
		\draw[e,->,black!40!red,very thick] (u6) to (u3);			
		\draw[e,->,black!40!red,very thick] (u7) to (u3);					
		\draw[e,->,black!40!red,very thick] (u9) to (u5);
	
		\draw[e,->,black!40!red,very thick] (w3) to (w1);			
		\draw[e,->,black!40!red,very thick] (w4) to (w2);					
		\draw[e,->,black!40!red,very thick] (w5) to (w2);							

		\def\rr{0.3}
		\draw[gy] (x3) to ++(100:\rr);
		\draw[gy] (x3) to ++(150:\rr);
		\draw[gy] (x3) to ++(-30:\rr);			
		\draw[gy] (x3) to ++(40:\rr);					
		\draw[gy] (x3) to ++(-120:\rr);					
		\draw[gy] (x4) to ++(45:\rr);
		\draw[gy] (x4) to ++(170:\rr);
		\draw[gy] (x4) to ++(-70:\rr);	
		\draw[gy] (x4) to ++(-135:\rr);
		\draw[gy] (x4) to ++(-10:\rr);
		\draw[gy] (x4) to ++(110:\rr);			
		\draw[gy] (x2) to ++(80:\rr);
		\draw[gy] (x2) to ++(173:\rr);
		\draw[gy] (x2) to ++(10:\rr);				
		\draw[gy] (x2) to ++(-80:\rr);
		\draw[gy] (x2) to ++(-105:\rr);						
		\draw[gy] (x2) to ++(130:\rr);								
		\draw[gy] (x1) to ++(30:\rr);
		\draw[gy] (x1) to ++(130:\rr);
		\draw[gy] (x1) to ++(-120:\rr);			
		\draw[gy] (x1) to ++(-50:\rr);			
		\draw[gy] (x1) to ++(85:\rr);		
		\draw[gy] (x7) to ++(160:\rr);
		\draw[gy] (x7) to ++(40:\rr);
		\draw[gy] (x7) to ++(95:\rr);		
		\draw[gy] (x7) to ++(-70:.8*\rr);		
		\draw[gy] (x7) to ++(-110:.8*\rr);		
		\draw[gy] (x5) to ++(120:\rr);			
		\draw[gy] (x5) to ++(-160:\rr);
		\draw[gy] (x5) to ++(-10:\rr);						
		\draw[gy] (x5) to ++(65:\rr);								
		\draw[gy] (x5) to ++(-75:\rr);								
		\draw[gy] (x8) to ++(140:\rr);
		\draw[gy] (x8) to ++(80:\rr);						
		\draw[gy] (x8) to ++(170:\rr);		
		\draw[gy] (x8) to ++(-130:\rr);						
		\draw[gy] (x8) to ++(-80:\rr);								
		\draw[gy] (x8) to ++(-20:\rr);								
		\draw[gy] (x9) to ++(170:\rr);
		\draw[gy] (x9) to ++(30:\rr);		
		\draw[gy] (x9) to ++(-80:\rr);			
		\draw[gy] (x9) to ++(85:\rr);
		\draw[gy] (x9) to ++(-30:\rr);				
		\draw[gy] (x9) to ++(-150:\rr);		
		
		\pgfmathsetseed{3}		
		\tikzset{snow/.style={circle,inner sep=0pt, minimum size=3pt}}
		\begin{pgfonlayer}{bg}
			\foreach \j in {1,...,5} {
				\pgfmathsetmacro{\x}{-3.5 + \j + rnd}; 
				\foreach \i in {1,...,25} {
					\node[snow, fill=blue!10, minimum size={(3+rand)*1pt}]
					     at (\x + rand * 0.1 ,{-1+(-\i+rand)*0.1}) {};
				}
			}
		\end{pgfonlayer}
		\foreach \j in {1,...,4} {
			\pgfmathsetmacro{\x}{-3 + \j*1.5 + rand}; 
			\foreach \i in {1,...,25} {
				\node[snow, fill=blue!20, minimum size={(3+rand)*1pt}]
					at (\x + rand * 0.1 ,{(-\i+rand)*0.1}) {};
			}
		}				
		
		\begin{pgfonlayer}{bg}		
			\draw[draw=brown!30,fill=brown!70] (-2.32,-4) rectangle ++(0.7,-1);
			\draw[draw=brown!30,fill=brown!70] (0.2,-4) rectangle ++(0.5,-1);			
			\draw[draw=brown!30,fill=brown!70] (1.8,-3) rectangle ++(0.4,-1.5);						
		\end{pgfonlayer}
	\fi
	}	
	\end{tikzpicture}
\end{center}
\end{frame}

\end{document}
